\newpage
\section*{\hfil ВВЕДЕНИЕ \hfil}
	В современной добыче полезных ископаемых активно применяются средства математического моделирования для симуляции процессов, происходящих в пласте. Однако, на данный момент полноценное моделирование не представляется возможным, поэтому используются различные приближенные методы. Для правильной симуляции процессов, происходящих в недрах, необходимо смоделировать саму среду, в которой эти процессы протекают. Обычно, данные о структуре среды доступны только в некотором количестве точек(скважин), в которых непосредственно идет добыча. Предлагается попытаться использовать методы машинного обучения для синтеза модели среды, похожей на природную. В рамках решения этой проблемы возникает задача синтеза текстур с трендами, например, для учета межскважинных корреляций.
	
	Известные работы в области синтеза текстур с помощью искуственных нейросетей \cite{texture-synthesis-using-CNN, texture-networks} показывают, что у нейросетевых моделей есть проблемы с воспроизведением регулярных или иных пространственно скоррелированных структур. Цель данной работы состоит в поиске и проверке нейросетевых архитектур, способных улавливать протяженные корреляции.