\section{Результаты}
	Описанные подходы были реализованы в виде компьютерных программ на языке Python с помощью библиотеки для построения искусственных нейронных сетей Keras \cite{keras}, который, в свою очередь, привлекает для расчетов библиотеку Tensorflow \cite{tf}. Использованные версии программных пакетов указаны в Приложении 1. Обучение проводилось на синтетических данных, сгенерированных самостоятельно разработанным генератором. 
	\subsection{Выборка с одним трендом}
		Выборка состояла из 6000 обучающих изображений и 316 валидационных. Все изображения содержали в себе различные случайные реализации одного тренда интенсивности
		$$тут формула \lambda(x) = $$
		тут картинки примеров из обучающей выборки
		\subsubsection{GAN}
		\subsubsection{Синтез текстур}
	\subsection{Выборка с множеством трендов}
		\subsubsection{GAN}
		\subsubsection{Синтез текстур}