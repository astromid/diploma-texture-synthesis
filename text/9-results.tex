\section{Результаты}
	Описанные подходы были реализованы в виде компьютерных программ на языке Python с помощью библиотеки для построения искусственных нейронных сетей Keras \cite{keras}, которая, в свою очередь, использует для расчетов библиотеку Tensorflow \cite{tf}. Использованные версии программных пакетов указаны в Приложении 1. Обучение проводилось на синтетических данных, сгенерированных самостоятельно реализованным генератором. 
	\subsection{Выборки с реализацией одного тренда}
		\subsubsection{Выборка 1}
			% sand:trend 2
			Выборка состояла из 3000 обучающих изображений и 500 валидационных. Все изображения содержали в себе различные случайные реализации одного тренда интенсивности
			$$тут формула \lambda(x) = 0.2 + 0.01875x : \lambda_i = 0.2, \quad \lambda_f = 5$$
			тут картинки примеров из обучающей выборки \\
			тут сгенерированные семплы \\
			тут метрики, графики
		\subsubsection{Выборка 2}
			% sand:trend 8
			Выборка состояла из 6000 обучающих изображений и 316 валидационных. Все изображения содержали в себе различные случайные реализации одного тренда интенсивности
			$$тут формула \lambda(x) = 1 + 0.0546875x: \lambda_i = 1, \quad \lambda_f = 15$$
			тут картинки примеров из обучающей выборки \\
			тут сгенерированные семплы \\
			тут метрики, графики
		\subsubsection{Выборка 3}
			% dust:trend 1
			Выборка состояла из 6000 обучающих изображений и 316 валидационных. Все изображения содержали в себе различные случайные реализации одного тренда интенсивности
			$$тут формула \lambda(x) = 1 + 0.19140625x: \lambda_i = 1, \quad \lambda_f = 50$$
			тут картинки примеров из обучающей выборки \\
			тут сгенерированные семплы \\
			тут метрики, графики
		
	\subsection{Выборки с реализацией различных трендов}
		\subsubsection{Выборка 4}
			% dust: trend 2
			Выборка состояла из 6000 обучающих изображений и 316 валидационных. Все изображения содержали в себе различные случайные реализации различных случайных трендов интенсивности $\lambda(x)$ \\
			тут картинки примеров из обучающей выборки \\
			тут сгенерированные семплы \\
			тут метрики, графики